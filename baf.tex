

\section{2. Theoretical Foundations of UIDT}
\subsection{2.1 The Information-Density Field}
\label{igw}

UIDT postulates that the information density field \( S(x) \) is the primary ontological entity, with dimensions


\[
[S] = \left[ \frac{k_B}{\ell^3} \right],
\]


where \( k_B \) is Boltzmann’s constant and \( \ell \) is a characteristic length scale. The field satisfies a Lorentz-covariant action:


\[
\mathcal{S} = \int d^4x \, \sqrt{-g} \left[ \frac{1}{2} \gamma \ell_P^2 (\nabla_\mu S \nabla^\mu S) - V(S) + \mathcal{L}_{\text{int}} \right],
\]


where \( \ell_P \) is the Planck length, \( \gamma \) is a dimensionless coupling, and the potential is given by


\[
V(S) = \frac{1}{2} m_S^2 S^2 + \frac{\eta}{4} S^4.
\]


The interaction term


\[
\mathcal{L}_{\text{int}} = \lambda S \, \mathrm{Tr}(F_{\mu\nu} F^{\mu\nu})
\]


couples \( S \) to Yang-Mills fields. The vacuum expectation value \( \langle S \rangle = S_0 \) breaks symmetries and generates masses.

The core equation of UIDT, derived from variation of the action, is


\[
m_{\text{eff}}^2 = m^2 + \gamma \frac{k_B^2}{c^4} \nabla_\mu S \nabla^\mu S,
\]


where the gradient term induces effective masses from information fluctuations.

