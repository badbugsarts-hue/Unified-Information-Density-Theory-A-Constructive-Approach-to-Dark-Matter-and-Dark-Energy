\section{Results and Observational Validation}

\subsection{DESI Data Release 2 and the Dark Energy Equation of State}

The DESI Data Release 2 (DR2), published in March 2025, provides the most precise measurement to date 
of the cosmic expansion history across the redshift range \(0.1 < z < 2.3\) \citep{DESI2025}. 
The survey measured baryon acoustic oscillations (BAO)—a characteristic scale imprinted in the distribution of galaxies 
by sound waves in the early universe—using spectroscopy of 14.3 million galaxies, quasars, and Lyman-\(\alpha\) forest systems. 
The BAO scale serves as a “standard ruler” that can be used to infer the expansion rate \(H(z)\) 
and angular diameter distance \(D_A(z)\) as functions of redshift.

When the DESI team fits their measurements with a \(\Lambda\)CDM model, they find excellent agreement with Planck CMB predictions, 
yielding \(H_0 = 67.97 \pm 0.38\) km/s/Mpc \citep{Planck2020}. 
However, when they allow for dynamical dark energy using the CPL parametrization 
\(w(z) = w_0 + w_a z/(1+z)\), they find a statistically significant preference for \(w_a \neq 0\). 
Specifically, combining DESI BAO with Planck CMB and Pantheon+ supernovae \citep{Brout2022}, they report:


\[
w_0 = -0.95 \pm 0.04, \quad w_a = -0.50 \pm 0.12,
\]


with the \(w_0 w_a\)CDM model preferred over flat \(\Lambda\)CDM at the \(4.2\sigma\) level based on DESI data alone, 
strengthening to \(6.4\sigma\) when combined with CMB and supernovae.

The UIDT predictions derived in Section~3 are:


\[
w_0 = -0.96 \pm 0.03, \quad w_a = -0.485 \pm 0.008,
\]


in remarkable agreement with the DESI measurements. 
The central values differ by only 0.01 for \(w_0\) and 0.015 for \(w_a\), 
well within the combined theoretical and observational uncertainties. 
The UIDT achieves this agreement without post-hoc parameter tuning; 
the values follow directly from the Barrow–Tsallis entropy formula with \(\Delta = 0.0037\) and \(\delta = 0.1191\).

Figure~1 (to be generated) will show the reconstructed \(w(z)\) from DESI data using Gaussian process regression 
compared to the UIDT prediction as a function of redshift. 
The plot demonstrates that UIDT tracks the data remarkably well across the entire redshift range, 
including the predicted phantom crossing at \(z_c \approx 0.5\).

The DESI collaboration also measured the expansion rate at specific effective redshifts. 
At \(z_{\text{eff}} = 0.51\), they find \(H(z) = 95.2 \pm 1.1\) km/s. 
The UIDT prediction at this redshift is:


\[
H(z=0.51) = 95.5 \pm 0.9 \,\text{km/s},
\]


representing a difference of just 0.3 km/s, or 0.3\%. 
This level of agreement, achieved across more than twenty independent redshift bins, 
provides strong validation of the UIDT expansion history.

\subsection{Euclid Quick Release 1 and Small-Scale Structure}

The Euclid Quick Release 1 (Q1), published in November 2025, represents the first scientific results from this ESA mission 
designed to map the geometry and structure of the universe with unprecedented precision \citep{Euclid2025}. 
The Q1 release focused on deep field observations in four regions totaling 63.1 square degrees. 
The visible-wavelength imaging achieved limiting magnitudes of \(m_{\text{AB}} \sim 27\), 
while the near-infrared imaging and spectroscopy enabled photometric and spectroscopic redshift measurements for millions of galaxies.

For our analysis, the most relevant Q1 results concern the dwarf galaxy population in the Perseus cluster and other nearby systems. 
As discussed in Section~4.2, Euclid identified 2,674 dwarf galaxy candidates with stellar masses between \(10^6\) and \(10^9\) solar masses. 
The morphological analysis revealed:
\begin{itemize}
    \item 58\% dwarf ellipticals (dE) with smooth, regular morphology,
    \item 42\% dwarf irregulars (dIrr) with clumpy, asymmetric structure,
    \item 53\% of dE show a central nucleus,
    \item 26\% of all dwarfs are associated with rich globular cluster systems.
\end{itemize}

These statistics are in excellent agreement with the UIDT predictions based on the thermodynamic clumping parameter \(\xi = 0.445\). 
The high fraction of ellipticals, the presence of nuclei, and the survival of globular clusters all support the cored dark matter halos 
predicted by UIDT’s thermodynamic damping mechanism, as opposed to the cuspy halos predicted by cold dark matter simulations.

Furthermore, the Euclid team measured the two-point correlation function of dwarf galaxies, 
finding a correlation length of \(r_0 = 4.2 \pm 0.3\) Mpc. 
In UIDT, this enhanced clustering arises because the information density field responds to existing matter concentrations 
through the \((1 + \xi \rho_m/\rho_{c0})\) factor. 
Our calculation predicts \(r_0 = 4.4 \pm 0.2\) Mpc, in good agreement with the observations.

The \(S_8\) parameter derived from Euclid Q1 weak lensing measurements is:


\[
S_8^{\text{Euclid}} = 0.758 \pm 0.005,
\]


consistent with the UIDT prediction of \(0.756 \pm 0.002\) and confirming the resolution of the \(S_8\) tension 
between CMB and weak lensing observations.
