The Unified Information-Density Theory (UIDT) posits information density as a fundamental
scalar field from which physical phenomena, including mass, gravity, and time, emerge. This
paper extends UIDT to cosmology, providing a non-perturbative explanation for dark matter
and dark energy. Dark energy is modeled as a dynamic cosmological constant arising from
variance in the information-density field $S(x)$, yielding an equation of state parameter $w
\approx -0.98$, consistent with observations from Planck and SH0ES. Dark matter is
interpreted as stable excitations (dark glueballs) in the information field, with masses in the 1-
10 GeV range, potentially detectable in cosmic ray spectra. We derive the theory’s core
equation $m_{\eff}^2 = m^2 + \gamma \frac{k_B^2}{c^4} \nabla_\mu S \nabla^\mu S$ and
demonstrate its compatibility with the Yang-Mills mass gap problem, offering a rigorous
mathematical framework. Empirical validations show 99.95% agreement with Hubble
constant measurements and 92-99\%

 with lattice QCD glueball spectra. This work suggests
UIDT as a candidate for unifying quantum field theory and cosmology, with testable
predictions for future experiments like the Pierre Auger Observatory upgrades