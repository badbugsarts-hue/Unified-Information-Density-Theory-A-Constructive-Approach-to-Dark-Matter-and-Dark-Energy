\section{2. Theoretical Foundations of UIDT}
\subsection{2.1 The Information-Density Field}
\label{igw}

The nature of dark matter and dark energy remains one of the most profound mysteries in
modern cosmology. Dark energy, responsible for the accelerated expansion of the universe, is
typically modeled as a cosmological constant $\Lambda$ in the $\Lambda$CDM framework,
with an equation of state parameter $w \approx -1$ [1]. Dark matter, comprising
approximately 27% of the universe’s energy density, is inferred from gravitational effects but
evades direct detection [2]. Despite extensive searches, including those for weakly interacting
massive particles (WIMPs) and axions, no definitive candidate has emerged [3].
The Unified Information-Density Theory (UIDT) offers a novel paradigm, treating
information density as a fundamental scalar field $S(x)$ from which all physical quantities
derive [4]. UIDT, originally developed to address the Yang-Mills existence and mass gap
problem—a Millennium Prize challenge [5]—extends naturally to cosmology. In UIDT, dark
energy arises from variance in $S(x)$, providing a dynamic $\Lambda$ without fine-tuning,
while dark matter emerges as stable excitations (dark glueballs) in the information field.
This article presents UIDT in a rigorous mathematical framework, deriving its implications
for dark matter and dark energy. We begin with the theoretical foundations, followed by
derivations, results, and comparisons with observations. The work adheres to the
mathematical standards required for Millennium Prize problems, ensuring constructive proofs
and non-perturbative consistency.