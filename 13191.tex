\section{Introduction}

The nature of dark matter and dark energy remains among the most profound mysteries in modern cosmology. 
Dark energy, responsible for the accelerated expansion of the universe first discovered through Type Ia supernova observations in 1998 
\citep{Riess1998,Perlmutter1999}, is typically modeled as a cosmological constant \(\Lambda\) in the standard \(\Lambda\)CDM framework, 
implying an equation of state parameter \(w \equiv p/\rho \approx -1\), 
where \(p\) represents pressure and \(\rho\) denotes energy density. 
This simple model has been remarkably successful in fitting observational data from the cosmic microwave background, 
large-scale structure surveys, and supernova distance measurements \citep{Planck2020}.

Dark matter, comprising approximately 27\% of the universe's energy density, 
is inferred from gravitational effects including galaxy rotation curves, gravitational lensing observations, 
and the formation of cosmic structure \citep{Clowe2006}, yet it evades direct detection in terrestrial experiments. 
Despite extensive searches spanning decades, no definitive particle candidate has emerged from experiments designed to detect 
weakly interacting massive particles, axions, or other hypothetical dark matter constituents \citep{Aprile2018}. 
The simultaneous presence of these two mysterious components---together accounting for 95\% of the universe's energy content---suggests 
that our understanding of fundamental physics remains incomplete.

Recent precision measurements from the Dark Energy Spectroscopic Instrument (DESI) Data Release 2, published in 2025, 
have revealed intriguing hints that dark energy may not behave as a pure cosmological constant 
\citep{DESI2025}. 
Analysis of baryon acoustic oscillation measurements across 14 million galaxies and quasars shows a preference at the 
\(2.8\sigma\) to \(4.2\sigma\) level for dynamically evolving dark energy with an equation of state that varies with redshift. 
Simultaneously, the Hubble tension---a persistent \(5\sigma\) discrepancy between the Hubble constant measured from the early universe 
using cosmic microwave background observations and from the late universe using the cosmic distance ladder---continues to resist resolution 
within the standard model \citep{Riess2022,Planck2020}. 
The \(S_8\) tension, a similar discrepancy between cosmic microwave background predictions and weak gravitational lensing measurements of matter clustering, 
adds further strain to the \(\Lambda\)CDM framework \citep{Heymans2021}.

The Unified Information-Density Theory (UIDT) offers a novel paradigm for addressing these observational tensions. 
Rather than treating information as an abstract descriptor of physical states, UIDT proposes that information density is the fundamental ontological entity, 
with conventional notions of mass, energy, and spacetime geometry emerging as derived quantities. 
This framework treats the information density field \(S(x)\), a Lorentz-covariant scalar field with mass dimension one, 
as the primary degree of freedom from which all physical phenomena derive. 
The field couples non-perturbatively to the Yang--Mills gauge field of quantum chromodynamics, 
providing a constructive solution to the Yang--Mills existence and mass gap problem---one of the seven Millennium Prize Problems in mathematics 
\citep{Faria2025}.

The theory was originally developed to address the Yang--Mills mass gap problem but extends naturally to cosmology 
through the introduction of a single dimensionless invariant \(\gamma \approx 16.339\). 
This invariant emerges from the self-consistent solution of three coupled nonlinear equations: 
a vacuum equation derived from extremizing the effective potential of the scalar field, 
a mass gap equation obtained from Schwinger--Dyson analysis of gluon propagators, 
and a renormalization group fixed point equation ensuring asymptotic safety at ultraviolet scales. 
The value of \(\gamma\) is not a free parameter fitted to data but rather a mathematical consequence of requiring these three equations 
to possess a stable, physical solution.

In this framework, dark energy arises naturally from the variance of the information density field \(S(x)\) across cosmic scales. 
The cosmological constant is not truly constant but proportional to the variance of \(S(x)\), 
which evolves with the expansion of the universe according to the thermodynamic properties of information encoded on the cosmic horizon. 
We implement this through Barrow--Tsallis hybrid entropy \citep{Barrow2020,Tsallis2009}, 
a generalization of the Bekenstein--Hawking entropy formula that incorporates both fractal quantum gravity corrections parametrized by a Barrow index \(\Delta\) 
and non-extensive statistical mechanics effects captured by a Tsallis parameter \(\delta\). 
Dark matter emerges as stable topological excitations in the information density field, 
analogous to solitonic solutions in classical field theory but stabilized by information-theoretic entropy constraints 
rather than conventional potential energy barriers.

This article presents UIDT in a rigorous mathematical framework, deriving its implications for dark matter and dark energy from first principles. 
We begin with the theoretical foundations in Section~2, establishing the action principle for the information density field 
and its coupling to gravity and gauge fields. Section~3 derives the dark energy predictions of UIDT, 
including the dynamical equation of state and its observational signatures in baryon acoustic oscillation measurements 
and cosmic microwave background observations. Section~4 addresses dark matter, presenting the theory of information-bound states 
and their phenomenology in galaxy dynamics and large-scale structure formation. Section~5 describes our computational methods, 
including the physics-informed neural network approach used to solve the nonlinear field equations and perform Bayesian parameter estimation 
against observational data. Section~6 presents our results, including detailed comparisons with DESI Data Release 2, 
Euclid Quick Release 1 \citep{Euclid2025}, JWST distance measurements \citep{Yan2025}, and ACT polarization data \citep{Naess2025}. 
We conclude in Section~7 with a discussion of falsifiable predictions testable in upcoming observational campaigns and experiments.

The work adheres to the mathematical standards required for Millennium Prize problems, 
ensuring constructive proofs and non-perturbative consistency throughout. 
All numerical calculations are fully reproducible, with open-source Python implementations provided in the supplementary materials. 
Our goal is to present UIDT not merely as a phenomenological model but as a fundamental theory of physics, 
subject to rigorous mathematical consistency checks and empirical falsification tests.
