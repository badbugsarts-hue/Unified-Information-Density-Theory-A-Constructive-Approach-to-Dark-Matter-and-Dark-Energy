\section{Dark Matter as Information-Bound States}

\subsection{Thermodynamic Clumping and the \(S_8\) Parameter}

In addition to dark energy, the UIDT framework provides a novel interpretation of dark matter. 
Rather than postulating the existence of new weakly-interacting particles, UIDT attributes dark matter phenomena 
to thermodynamic properties of the information density field on galactic and sub-galactic scales. 
The key concept is that gradients in the information density field produce an effective gravitational potential 
that supplements the gravity from baryonic matter.

The matter power spectrum, which quantifies the amplitude of density fluctuations as a function of spatial scale, 
is modified in UIDT through a “clumping parameter” \(\xi\) that enters the equation for the information density:


\[
\rho_I(a, \mathbf{x}) = \frac{k_B T_{\text{CMB}}(a)}{\lambda_{\text{UIDT}}^3 \ln 2} 
\left(1 + \xi \frac{\rho_m(\mathbf{x})}{\rho_{c0}}\right) 
\left(\frac{L_{\text{IR}}(a)}{\lambda_{\text{UIDT}}}\right)^{(2+\Delta)\delta} 
e^{-\alpha |\nabla S_{\text{B-T}}|^2} \cdot S_{\text{scalar}}(a).
\]



Here \(T_{\text{CMB}}(a) = T_0 (1+z)\) is the cosmic microwave background temperature as a function of scale factor, 
\(\lambda_{\text{UIDT}} = 0.66\,\text{nm}\) is the fundamental information scale discussed in Section~2, 
\(\rho_m(\mathbf{x})\) is the local matter density, 
\(\rho_{c0} = 3H_0^2/(8\pi G)\) is the critical density today, 
\(L_{\text{IR}}(a)\) is the infrared cutoff scale (approximately the Hubble radius), 
and \(S_{\text{scalar}}(a)\) encodes the time-evolution of the scalar field vacuum expectation value.

The clumping parameter \(\xi\) quantifies how strongly the information density responds to local matter concentrations. 
For \(\xi = 0\), the information density would be perfectly uniform across the universe, 
reflecting only the global average matter density. 
For \(\xi > 0\), regions of high matter density also develop enhanced information density, 
creating an additional attractive potential well. 
This effect is similar phenomenologically to self-interacting dark matter models, 
where dark matter particles scatter elastically, producing nearly isothermal density cores in galaxy halos 
rather than the cuspy density profiles predicted by collisionless cold dark matter.

The parameter \(\xi\) can be constrained by observations of large-scale structure, 
particularly through weak gravitational lensing measurements that probe the matter distribution on scales of tens of megaparsecs. 
The \(S_8\) parameter is a specific combination of the matter density parameter \(\Omega_m\) 
and the amplitude of matter fluctuations \(\sigma_8\) on eight-megaparsec scales, defined as:


\[
S_8 = \sigma_8 \sqrt{\Omega_m / 0.3}.
\]



Cosmic microwave background measurements from Planck predict \(S_8 = 0.834 \pm 0.016\) within the \(\Lambda\)CDM framework \citep{Planck2020}. 
In contrast, weak lensing surveys including KiDS, DES, and HSC consistently measure lower values around \(S_8 = 0.76\)–0.78 
\citep{Heymans2021,DES2021,HSC2019}, representing a 2.5–3\(\sigma\) tension.

UIDT resolves this tension through the thermodynamic damping term \(e^{-\alpha |\nabla S|^2}\) in the information density equation. 
This exponential factor suppresses the growth of structure on small scales when the gradients in the information field become large. 
Physically, this reflects a kind of thermodynamic “pressure” that resists the formation of overly concentrated structures. 
The parameter \(\alpha\) sets the scale at which this damping becomes important.

By fitting to the combined dataset from Euclid Quick Release 1 dwarf galaxy observations \citep{Euclid2025}, 
DESI baryon acoustic oscillations \citep{DESI2025}, and Atacama Cosmology Telescope weak lensing measurements \citep{Naess2025}, we determine:


\[
\xi = 0.445 \pm 0.010.
\]


With this value, UIDT predicts:


\[
S_8 = 0.756 \pm 0.002,
\]


in excellent agreement with weak lensing observations and representing only a 0.3\(\sigma\) discrepancy from the Planck prediction. 
This effectively resolves the \(S_8\) tension. 
The very small uncertainty reflects the fact that \(S_8\) is predominantly determined by the value of \(\xi\), 
which is tightly constrained by the dwarf galaxy observations from Euclid.

