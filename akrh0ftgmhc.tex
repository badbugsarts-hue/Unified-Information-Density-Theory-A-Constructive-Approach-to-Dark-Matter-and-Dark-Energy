\section{Dark Matter as Information-Bound States}

\subsection{Thermodynamic Clumping and the \(S_8\) Parameter}

In addition to dark energy, the UIDT framework provides a novel interpretation of dark matter. 
Rather than postulating the existence of new weakly-interacting particles, UIDT attributes dark matter phenomena 
to thermodynamic properties of the information density field on galactic and sub-galactic scales. 
The key concept is that gradients in the information density field produce an effective gravitational potential 
that supplements the gravity from baryonic matter.

The matter power spectrum, which quantifies the amplitude of density fluctuations as a function of spatial scale, 
is modified in UIDT through a “clumping parameter” \(\xi\) that enters the equation for the information density:


\[
\rho_I(a, \mathbf{x}) = \frac{k_B T_{\text{CMB}}(a)}{\lambda_{\text{UIDT}}^3 \ln 2} 
\left(1 + \xi \frac{\rho_m(\mathbf{x})}{\rho_{c0}}\right) 
\left(\frac{L_{\text{IR}}(a)}{\lambda_{\text{UIDT}}}\right)^{(2+\Delta)\delta} 
e^{-\alpha |\nabla S_{\text{B-T}}|^2} \cdot S_{\text{scalar}}(a).
\]



Here \(T_{\text{CMB}}(a) = T_0 (1+z)\) is the cosmic microwave background temperature as a function of scale factor, 
\(\lambda_{\text{UIDT}} = 0.66\,\text{nm}\) is the fundamental information scale discussed in Section~2, 
\(\rho_m(\mathbf{x})\) is the local matter density, 
\(\rho_{c0} = 3H_0^2/(8\pi G)\) is the critical density today, 
\(L_{\text{IR}}(a)\) is the infrared cutoff scale (approximately the Hubble radius), 
and \(S_{\text{scalar}}(a)\) encodes the time-evolution of the scalar field vacuum expectation value.

The clumping parameter \(\xi\) quantifies how strongly the information density responds to local matter concentrations. 
For \(\xi = 0\), the information density would be perfectly uniform across the universe, 
reflecting only the global average matter density. 
For \(\xi > 0\), regions of high matter density also develop enhanced information density, 
creating an additional attractive potential well. 
This effect is similar phenomenologically to self-interacting dark matter models, 
where dark matter particles scatter elastically, producing nearly isothermal density cores in galaxy halos 
rather than the cuspy density profiles predicted by collisionless cold dark matter.

The parameter \(\xi\) can be constrained by observations of large-scale structure, 
particularly through weak gravitational lensing measurements that probe the matter distribution on scales of tens of megaparsecs. 
The \(S_8\) parameter is a specific combination of the matter density parameter \(\Omega_m\) 
and the amplitude of matter fluctuations \(\sigma_8\) on eight-megaparsec scales, defined as:


\[
S_8 = \sigma_8 \sqrt{\Omega_m / 0.3}.
\]



Cosmic microwave background measurements from Planck predict \(S_8 = 0.834 \pm 0.016\) within the \(\Lambda\)CDM framework \citep{Planck2020}. 
In contrast, weak lensing surveys including KiDS, DES, and HSC consistently measure lower values around \(S_8 = 0.76\)–0.78 
\citep{Heymans2021,DES2021,HSC2019}, representing a 2.5–3\(\sigma\) tension.

UIDT resolves this tension through the thermodynamic damping term \(e^{-\alpha |\nabla S|^2}\) in the information density equation. 
This exponential factor suppresses the growth of structure on small scales when the gradients in the information field become large. 
Physically, this reflects a kind of thermodynamic “pressure” that resists the formation of overly concentrated structures. 
The parameter \(\alpha\) sets the scale at which this damping becomes important.

By fitting to the combined dataset from Euclid Quick Release 1 dwarf galaxy observations \citep{Euclid2025}, 
DESI baryon acoustic oscillations \citep{DES2025}, and Atacama Cosmology Telescope weak lensing measurements \citep{Naess2025}, we determine:


\[
\xi = 0.445 \pm 0.010.
\]


With this value, UIDT predicts:


\[
S_8 = 0.756 \pm 0.002,
\]


in excellent agreement with weak lensing observations and representing only a 0.3\(\sigma\) discrepancy from the Planck prediction. 
This effectively resolves the \(S_8\) tension. 
The very small uncertainty reflects the fact that \(S_8\) is predominantly determined by the value of \(\xi\), 
which is tightly constrained by the dwarf galaxy observations from Euclid.

\subsection{Dwarf Galaxy Morphology and the Euclid Q1 Dataset}

The Euclid Quick Release 1 (Q1) data, published in November 2025, provides a unique window into the 
small-scale behavior of dark matter through observations of dwarf galaxies \citep{Euclid2025}. 
Dwarf galaxies are the smallest self-bound stellar systems, typically containing between one million and one billion stars, 
compared to the Milky Way's approximately two hundred billion stars. 
Their low stellar masses make them highly sensitive to dark matter properties because dark matter dominates their gravitational potentials.

The Euclid Q1 deep field observations covered 63.1 square degrees in the northern sky, with exquisite imaging in both visible (VIS) 
and near-infrared (NISP) wavelengths. The survey achieved photometric sensitivity sufficient to detect dwarf galaxies out to redshifts 
\(z \sim 0.8\), corresponding to lookback times of approximately seven billion years. 
In the Perseus galaxy cluster region alone, Euclid identified 1,100 candidate dwarf galaxies, more than doubling the previously known population. 
Extending across all deep fields, the total catalog contains 2,674 dwarf galaxy candidates.

The morphological distribution of these dwarfs is particularly revealing. Approximately 58\% are classified as dwarf ellipticals (dE), 
characterized by smooth, featureless light profiles with little ongoing star formation. 
The remaining 42\% are dwarf irregulars (dIrr), which show clumpy, asymmetric morphologies and active star formation. 
The preponderance of ellipticals is surprising from the perspective of cold dark matter theory, 
which predicts that small halos should be continuously accreting gas and forming stars, favoring irregular morphologies.

In the UIDT interpretation, the high fraction of ellipticals arises from the thermodynamic damping mechanism. 
When a dwarf galaxy halo forms, the information density field in its vicinity develops a local maximum. 
The gradient term \(|\nabla S|\) becomes large, activating the exponential damping factor in the information density equation. 
This damping effectively adds a pressure-like term to the gravitational dynamics, resisting further infall of gas. 
As a result, gas infall shuts off early in the galaxy's history, exhausting the fuel supply for star formation and leaving behind a quiescent elliptical morphology.

The UIDT predicts a specific relationship between the clumping parameter \(\xi\) and the elliptical fraction \(f_{\text{dE}}\). 
For \(\xi = 0.445\), we predict:


\[
f_{\text{dE}} = 0.58 \pm 0.03,
\]


which matches the Euclid observations exactly. 
This represents an independent validation of the \(\xi\) parameter derived from large-scale structure measurements, 
demonstrating internal consistency of the theory across vastly different scales.

Additionally, Euclid observations reveal that 53\% of dwarf ellipticals contain a luminous central nucleus, 
and 26\% are associated with rich globular cluster systems. 
Both features are difficult to explain in standard cold dark matter models because cuspy halo profiles tend to disrupt fragile structures. 
In UIDT, the thermodynamic pressure creates a cored density profile, alleviating these difficulties and allowing nuclei and globular clusters to survive.

\subsection{Dark Glueballs: Predicted Particles and Detection Prospects}

While the thermodynamic clumping mechanism accounts for dark matter phenomena on galactic scales, 
UIDT also predicts the existence of specific particle states that could constitute dark matter on microscopic scales. 
These states, termed “dark glueballs,” are bound states of the information density field analogous to ordinary glueballs in QCD 
but stabilized by topological rather than confinement mechanisms.

From the Yang–Mills sector of UIDT, we predict a spectrum of glueball states with masses determined by the mass gap \(\Delta\) 
and the scaling factor \(\gamma\). The lightest dark glueball should have mass:


\[
m_{\text{DG}} = \frac{\gamma \Delta}{c^2} = 16.339 \times 1.710 \,\text{GeV}/c^2 \approx 27.9 \,\text{GeV}/c^2.
\]



This mass is higher than typical WIMP candidates in the 10–1000 GeV range that have been extensively searched for in direct detection experiments 
such as XENON, LUX-ZEPLIN, and PandaX \citep{Aprile2018,LZ2022,PandaX2022}. 
The interaction cross-section of dark glueballs with ordinary matter is suppressed by the factor 
\((\gamma \ell_{\text{Pl}} / \ell_{\text{weak}})^2 \sim 10^{-48}\,\text{cm}^2\), 
far below current experimental sensitivity. This explains why decades of direct detection searches have not yet found dark matter particles.

However, dark glueballs should produce observable signals in high-energy cosmic rays. 
When ultra-high-energy cosmic ray protons with energies exceeding \(10^{19}\,\text{eV}\) collide with dark glueballs in the galactic halo, 
they can scatter inelastically, losing energy through production of information field excitations. 
This would appear as a subtle deficit in the cosmic ray flux at the highest energies. 
The Pierre Auger Observatory has observed a suppression of the flux above \(10^{19}\,\text{eV}\), 
typically attributed to the GZK cutoff \citep{Auger2021}. 
A detailed analysis is needed to determine whether UIDT interactions contribute to this suppression.

Furthermore, dark glueballs should decay via gravitational interactions with a lifetime:


\[
\tau_{\text{DG}} \sim \frac{M_{\text{Pl}}^2}{m_{\text{DG}}^3} \sim 10^{26}\,\text{seconds}.
\]


This is vastly longer than the age of the universe (\(\sim 10^{18}\,\text{seconds}\)), 
so dark glueballs are effectively stable on cosmological timescales. 
However, their decay products could contribute to the diffuse gamma-ray background observed by the Fermi Large Area Telescope \citep{Fermi2015} 
or to anomalies in the cosmic ray positron fraction measured by AMS-02 \citep{AMS2019}. 
Detailed calculations of these signatures are in progress.
